\documentclass[a4paper, 14pt]{extarticle}
%\usepackage[14pt]{extsizes}
\usepackage[utf8]{inputenc}
\usepackage[T2A]{fontenc}
\usepackage[english,russian]{babel} 
\usepackage[left=25mm, top=20mm, right=25mm, bottom=20mm, nohead, nofoot]{geometry}
\usepackage{amsmath,amsfonts,amssymb} % математический пакет
\usepackage{setspace}
\usepackage{fancybox,fancyhdr} 
\usepackage{xcolor}
\usepackage{hyperref} 
\usepackage{graphicx}%Вставка картинок правильная
\usepackage{float}%"Плавающие" картинки
\usepackage{wrapfig}%Обтекание фигур (таблиц, картинок и прочего)
%\usepackage{indentfirst}%Красная строка

\pagestyle{fancy}
\fancyhf{}
%\fancyhead[R]{\href{https://github.com/MrShprotter}{Джамал Лазейкин}}
\fancyhead[C]{Государственная столичная гимназия}
\fancyfoot[R]{\thepage} 

\setcounter{page}{2} % счетчик нумерации страниц
\headsep=10mm 
\hypersetup{colorlinks=true, allcolors=[RGB]{010 090 200}} % цвет ссылок 
\newcommand{\lr}[1]{\left({#1}\right)} % команда для скобок
\setlength\parindent{17.0pt}

\begin{document}

\begin{titlepage}
\begin{center}
Департамент образования и науки города Москвы\\
ГБОУ «@@@@@@@»\\
\end{center}

%\vspace{8em}

\begin{figure}[h]
\centering
\includegraphics[width=0.4\linewidth]{media/logo33.png}
\end{figure}

\vspace{2em}

\begin{center}
\large\textsc{Проект\textbf{\linebreak «Fast Math»}}
\end{center}

\vspace{6em}

\begin{flushright}
\textbf{\textit{Выполнил:}}\\
@@@@@@@@@@@@@\\
@@@@@@@@@@@@@\\
@@@@@@@@@@@@@\\
\textbf{\textit{Научный руководитель:}}\\
@@@@@@@@@@@@@\\
@@@@@@@@@@@@@\\
\vspace{2em}
\underline{\hspace{14em}}
\end{flushright}

\vspace{6em}

\begin{center}
\textbf{Москва \\2023}
\end{center}

\end{titlepage}

\tableofcontents
\newpage

\section{Методологический паспорт}
\newpage

\section{Введение}
Проект Fast Math заключается в создании коротких образовательных видеороликов по математике, которые
содержат основную информацию по какой-либо теме. Они позваляют понять
тему на начальном уровне, что ускоряет её будущее изучение.
Видео подходят для просмотра на уроке и индивидуального просмотра.
Также они направлены на полное понимание тема, а не на заучивание.
Поэтому в видео объясняются некоторые вещи,
которые на обычном уроке могут быть поданы просто для запоминания.
Например, вывод формул и их связь с другими, геометрическия интерпретации.
Все видеоролики распространяются на видеохостинге YouTube\texttrademark
и на сайте проекта\footnote{
    К сожелению, на данный момент сайт не доступен
}.

\newpage
\section{Создание продукта}

\subsection{Инструменты}

\subsubsection{Manim}
Для создания роликов я использовал язык программирования Python
и модуль Manim. Он позваляет конвертировать
програмный код в видеоролики.
В Manim большинство инструментов заточены для визуализации математики.
Создал его блогер Грант Сандерсон, также известный как 3b1b\footnote{
    \href{3blue1brown.com}{Офицальный сайт}
}.
У Manim множество плюсов:

\begin{itemize}
    \item Простые и понятные названия элементов и функций
    \item Поддержка \LaTeX
    \item Распространение под свободной лицензией MIT\footnote{\href
            {https://github.com/ManimCommunity/manim/blob/main/LICENSE}
            {Текст лицензии}
        }
    \item Хорошая документация
    \footnote{
        \href{https://docs.manim.community/}{Ссылка н документацию}
    }
    \item Большое кол-во уроков по библиотеке
    \item CLI интерфейс пакета
\end{itemize}

Manim поддерживает использование сторонних SVG-файлов и работу с ними.
Для рендера видео могут импользоватся две библиотеки: OpenGl и Cairo.
OpenGl работает быстрее, но при это имеет неудобную работу с окнами,
плохой предпросмотр и некоторые другие проблемы, поэтому я использую Cairo.
Поскольку Manim сначала создаёт небольшие кусочки видео, их нужно соеденить,
это возможно сделать с помощью ffmpeg. Подробнее об этом я буду говорить
в пункте про видеоредакторы.
Несмотря на то, что Manim эффективный и быстрая модуль, оно имеет ряд
недостатков. Самый большой из них - это отсутствие поддержки кириллицы
в \LaTeX. Для его исправления необходимо внести изменения в код пакета.
Также у Manim есть две основные версии: Manim3b1b и ManimCommunity.
Manim3b1b разрабатывается только Грантом Сандерсоном, а Community версию
поддерживает сооющество. Я использую Community версию, поскольку
анимации в ней работают лучше.

\subsubsection{EManim}
Поскольку Manim не содержит некотрых функций, которые были мне необходимы,
мной был создан модуль Python EManim\footnote{Extendet Manim}.
Он включает в себя возможность автомотического вывода субтитров, уведмления,
начальную заставку и множество других функций. Для удобства установки
и обновления модуль загружен в PyPi и на данный момент доступен
для установки любым пользователем пакетного менеджера pip.
EManim распространяется под лицензией GNU GPL 3.0\footnote{
    \href{https://www.gnu.org/licenses/gpl-3.0.en.html}{Ссылка на лицензию}
}

\subsubsection{IDE}
Для работы с кодом неоходим хотябы редактор кода, а лучше использовать IDE
\footnote{
    Integrated Development Environment (Интегрированая Среда Разработки)
}.
В качестве данного иструмента я использовал Neovim, улучшенный форк Vim.
Посколько Neovim изначально не обладает основными функциями IDE,
нужно использовать плагины. В качестве менеджера плагинов я использовал
Packer и установил следующие плагины:
\begin{itemize}
    \item TreeSitter для подсветки синтаксиса
    \item Nvim-lsp для поддержки LSP серверов
    \item Nvim-cmp для автодополнения
    \item Lspkind
    \item Lualine
    \item Bufferline
\end{itemize}
Также я установил LSP сервер pyright. Все эти настройки позволили
значительно увеличить скорость разработки по стравнению с другими IDE,
например с VS Code.

\subsubsection{Git}
При разработке необходимо использовать систему контроля версий.
Это позволит контролировать изменения в продукте, значительно упростит
работу в команде и резервное копироване. Самая эфективная, быстрая, простая
и популярная из них - это Git. Git была разработана Линусом Торвальдсом для
поддержки ядра Linux. Эта VCS\footnote{
    Version Control System (Система Контроля Версий)
} является децентрелизованой и распространяется под свободной лицензией
GNU GPL 2. Для более эфективной работы желательно использовать Git-сервер,
куда будет дублирован репозиторий. Изначально я выбрал решение от компании
Microsoft-Github. Github удобная и популярная систему,
но есть несколько причин, из-за которых этот сайт является нежелательным
к использованию. Во-первых Github блокирует некоторые репозитории только
по причине происхождения разработчиков, например недавно пострадали многие
российские разработчики. Этот фактор не позваляет доверять Github.
Также Фонд Свободного ПО не одорил этот шаг со строны Microsoft и перестал
сотрудничать с компаниями, которые используют Github. Во-вторых, код сервера
Github не является октрытым, поэтому никто не может развернуть его на
своём сервере, дополнить или улучшить. По этим причинам я был вынужден
задуматься о переходе на другой хостинг.

Существует множество открытых аналогов Github, распространяемых под
свободной лицензией. Самые популярные из них: Gitlab, Gitea и Gogs.
Лучшим решением было бы развернуть свой собственный сервер и разместить на нём
репозиторий проект, но, к сожелению, моя рабочая станция не обладает
достаточной мощностью, а аренду виртуального сервера я не могу позволить
из-за отсутствия финансирования проекта. Поэтому мной был рассмотрен переход
на сторонний хостинг, основанный на одном из вышеперечисленных решений.
Среди многих вариантов я выбрал сайт codeberg.org.

\end{document}
